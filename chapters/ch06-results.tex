\chapter{Results}
\label{ch06:results}
This chapter presents the performance results of \ac{ART} and various Kafka replication scenarios. All values in the tables have been rounded to the nearest whole message, while percentages are rounded to three significant figures for consistency and accuracy.
% introduction to results, recap what experiments were run. percentages rounded to 3 sigfigs, everything else to nearest message.

\section{Target Adapter Performance}
\label{ch06:results:artperformance}
\ac{ART} turned out to be the slowest, on average, of all the scenarios measured. The average message throughput can be seen in Figure \ref{fig:chapter06:results:artavgmessage} and the specific values in Table \ref{tab:art:messagethroughput}. The message throughput lies at an average of 1.9K messages per second. The entire \ac{IS} process takes approximately 14 minutes to complete. There is some variation in the average message throughput.
% The throughput of each specific run can be found in Figure \ref{fig:appendix02:graphs:artallrunsmessage}.

\begin{table}
    \centering
    \begin{tabular}{|cc|}
        \hline
         \textbf{Average Throughput} & 1,934 msgs/sec \\
         \textbf{Median Throughput} & 1,945 msgs/sec \\
         \textbf{Maximum Throughput} & 3,136 msgs/sec \\
        \hline
    \end{tabular}
    \caption{Target Adapter Message Throughput Results}
    \label{tab:art:messagethroughput}
\end{table}

\begin{figure}[htbp]
    \centering
    \includegraphics[width=0.75\textwidth]{chapters/images/art-performance/art-avg-message-throughput.png}
    \caption{Target Adapter average message throughput}
    \label{fig:chapter06:results:artavgmessage}
\end{figure}

The visualization of the transaction throughput can be seen in Figure \ref{fig:chapter06:results:artavgtransaction}. It was presented as a scatter plot instead of a line graph due to the throughput measurement not being continuous. \ac{ART} processed transactions in bursts of typically 1-2 seconds, as can be seen in Figure \ref{fig:chapter06:results:arttransactionminute}, which shows an excerpt of \ac{ART}'s processed transactions during one of the experiment runs. To take into account these processing bursts, the results in Table \ref{tab:art:transactionthroughput} include the active average throughput, which ignores the burst intervals. This allows the differentiation of how many transactions are processed on average during the bursts, as opposed to the overall average throughput.
% The transaction throughput for each specific run can be found in Figure \ref{fig:appendix02:graphs:artallrunstransaction}.

\begin{table}
    \centering
    \begin{tabular}{|cc|}
        \hline
         \textbf{Average Throughput} & 4 transactions/sec \\
         \textbf{Active Average Throughput} & 10 transactions/sec \\
         \textbf{Median Active Throughput} & 10 transactions/sec \\
         \textbf{Maximum Active Throughput} & 12 transactions/sec \\
        \hline
    \end{tabular}
    \caption{Target Adapter Transaction Throughput Results}
    \label{tab:art:transactionthroughput}
\end{table}

\begin{figure}[htbp]
    \centering
    \includegraphics[width=0.75\textwidth]{chapters/images/art-performance/art-avg-transaction-throughput.png}
    \caption{Target Adapter average transaction throughput}
    \label{fig:chapter06:results:artavgtransaction}
\end{figure}

\begin{figure}[htbp]
    \centering
    \includegraphics[width=1\textwidth]{chapters/images/art-performance/art-transaction-throughput-minute.png}
    \caption[Metric collection excerpt of Target Adapter's processed transactions]{Metric collection excerpt of Target Adapter's processed transactions (see \ref{fig:appendix02:results:arttransactionminute} for enlarged version)}
    \label{fig:chapter06:results:arttransactionminute}
\end{figure}

\section{Kafka-Based Replication Performance}
\label{ch06:results:artperformance}
The Kafka Connect replication solution was tested with five different configurations detailed in \ref{ch05:methodology:design:scenarios}, each representing an increasing degree of parallelization. For improved legibility, each scenario was assigned a number.

\begin{description}
    \item [Scenario 1]
    This scenario represents the "One Task, One Partition" configuration, or the one with no parallelization. The source and sink connector are running one task each. The topic is also composed of only one partition, running on a single broker.

    \item[Scenario 2]
    This scenario represents the "Three Tasks on One Worker, Three Partitions" configuration. The source and sink connector are assigned three tasks each. There are two workers, one for each connector. The topic is split into three partitions, each hosted on a broker of its own.

    \item[Scenario 3]
    This scenario represents the "Seven Tasks on One Worker, Seven Partitions" configuration, representing a higher degree of parallelization than Scenario 2. Each connector was configured to run 7 brokers, with one worker per connector. There was also an increase to 7 topic partitions, hosted on a total of 3 brokers.

    \item[Scenario 4]
    This scenario represents the "Seven Tasks on Two Workers, Seven Partitions" configuration, which has the same degree of parallelization as Scenario 3, but with task distribution on two workers per connector.
    
    \item[Scenario 5]
    This scenario represents the "20 Tasks on Five Workers, 20 Partitions" configuration and was chosen to test how well the performance scales with a significant increase in tasks and partitions. It involves 20 tasks for the source and sink connector each. Each connector was run on five workers, and the topic was broken up into 20 partitions, with three brokers hosting the topic.
\end{description}

In general, throughput can be seen to grow with each increase in parallelization. The message throughput, visualized for the three different pipeline components in Figure \ref{fig:chapter06:results:allscenariosmessage}, increases visibly with each scenario. The source connector in Figure \ref{fig:chapter06:results:sourceallscenariosmessage} experiences its most significant improvement in message throughput in Scenario 4, compared to Scenario 3. The same can be said for its transaction throughput, shown in Figure \ref{fig:chapter06:results:sourceallscenariostrans}. The broker's message throughput in Figure \ref{fig:chapter06:results:brokerallscenarios} experiences a similar massive increase in message throughput in Scenario 4. This was unexpected, as the significant increase in parallelization actually happened in Scenario 5 with more than double the number of tasks and workers. The rest of the scenarios were similar in pattern to the source connector. The sink connector shown in Figure \ref{fig:chapter06:results:sinkallscenarios}, on the other hand, did not show the same performance increase in Scenario 4. Instead, the performance increase occurred as expected in Scenario 5, although with very high variability and multiple drops in performance. It is interesting to note that the source connector is more performant than the sink connector in Figure \ref{fig:chapter06:results:allscenariosmessage}, with a higher message throughput observed in all scenarios by comparison.

Both the sink and source connectors typically experienced a high variability in the average message throughput. This is especially noticeable for the source connector in Figure \ref{fig:chapter06:results:sourceallscenariosmessage}, additionally more detailed in Figure \ref{fig:appendix02:results:sourcemessages}, where more prevalent performance drops occur approximately every minute of the elapsed time. The sink connector experiences similar drops in performance, as can be seen in Figure \ref{fig:chapter06:results:sinkallscenarios}, as well as in more detail in Figure \ref{fig:appendix02:results:sinkmessages}. However, the drops typically occur at different intervals than those of the source connector, and in some scenarios more often (especially in Scenario 3 and Scenario 4). The average message throughput for the broker, on the other hand, is more stable. It has no noticeable drops in any scenario, as seen in Figure \ref{fig:chapter06:results:brokerallscenarios} and in more detail in Figure \ref{fig:appendix02:results:brokermessages}.
% also talk about beggining drops in transaction throughput
% add appendix showing averages with drops more defined

The more exact measurements, with the mean, median, and maximum throughput values, can be found in Table \ref{tab:kafka:messagethroughputsource} and \ref{tab:kafka:transactionthroughput} for the source connector, Table \ref{tab:kafka:messagethroughputbroker} for the broker, and Table \ref{tab:kafka:messagethroughputsink} for the sink connector.

The average message throughput of the source connector in Table \ref{tab:kafka:messagethroughputsource} increased by 374\% in Scenario 5 compared to the single-threaded Scenario 1, showcasing the extent to which parallelization impacted the connector's performance. The most significant improvement in performance, from Scenario 3 to Scenario 4, resulted in an increase of 80.8\%. The increase between Scenario 4 and 5, expected to be the most significant increase in performance between scenarios due to a much higher degree of parallelization, was only improved by 16.6\%. The average transaction throughput in Table \ref{tab:kafka:transactionthroughput} showed similar improvements. There was an increase of 314\% between Scenarios 1 and 5, while the most significant improvement between the scenarios resulted in an increase of 81.5\%.

The average message throughput of the broker increased by 290\% from Scenario 1 to Scenario 5, also showing the effect of increased parallelization on performance. Similarly to the source connector, the most noticeable increase occurred from Scenario 3 to Scenario 4, measuring an improvement of 71.5\%. The performance increase between Scenario 4 and Scenario 5 was at 16\%. Generally, the broker message throughput changed at approximately the same rate as the source connector. This can also be noticed visually when comparing Figures \ref{fig:chapter06:results:brokerallscenarios} and \ref{fig:chapter06:results:sourceallscenariosmessage}.

The sink connector's average message throughput, found in Table \ref{tab:kafka:messagethroughputsink}, diverged significantly from that of the broker and source connector. First of all, the sink connector had significantly lower overall performance compared to the source connector and broker. In Scenario 1, the sink connector's average message throughput is 47.9\% lower than the broker's average message throughput. The most significant increase from scenario to scenario was also different, occurring from Scenario 4 to 5 instead with an increase in 78.9\%. The total increase from Scenario 1 to Scenario 5 was by 514\%, representing by far the largest increase compared to the source connector and broker. 

Another observation of high variability can be seen in the greater difference between the mean and median average message throughput in each scenario in Table \ref{tab:kafka:messagethroughputsource} for the source connector and Table \ref{tab:kafka:messagethroughputsink} for the sink connector. The median in all scenarios is typically higher than the mean, indicating a possible skewed distribution of the data. A possible contribution to that could be the performance drops, both from the low performance when just starting the \ac{IS} and the dropping performance at the very end.

Meanwhile, the broker throughput in Figure \ref{fig:chapter06:results:brokerallscenarios} visibly scales in performance together with the source connector. This is the case even in Scenarios 3, 4, and 5, where each broker hosts multiple partitions. However, the average throughput in Table \ref{tab:kafka:messagethroughputbroker} only exceeds the source connector in Scenarios 1 and 2. In Scenario 3 the broker's performance is by 2.8\% less than the source connector's, in Scenario 4 by 7.8\%, and in Scenario 5 by 8.2\%.

% later scenarios processed IS very fast, not a lot of data points

\begin{figure}[htbp]
    \centering
    \includegraphics[width=0.75\textwidth]{chapters/images/allscenarios/source-avg-runs-all-scenarios-transaction.png}
    \caption{Source connector average transaction throughput}
    \label{fig:chapter06:results:sourceallscenariostrans}
\end{figure}

\begin{figure}[htbp]
    \centering
    \subfloat[Source connector]{\label{fig:chapter06:results:sourceallscenariosmessage}
        \centering
        \includegraphics[width=0.75\textwidth]{chapters/images/allscenarios/source-avg-runs-all-scenarios.png}
    }
    \hfill
    \subfloat[Broker]{\label{fig:chapter06:results:brokerallscenarios}
        \centering
        \includegraphics[width=0.75\textwidth]{chapters/images/allscenarios/broker-avg-runs-all-scenarios.png}
    }
    \hfill
    \subfloat[Sink connector]{\label{fig:chapter06:results:sinkallscenarios}
        \centering
        \includegraphics[width=0.75\textwidth]{chapters/images/allscenarios/sink-avg-runs-all-scenarios.png}
    }
    \hfill
    \caption{Average message throughput of various Kafka replication scenarios}
    \label{fig:chapter06:results:allscenariosmessage}
\end{figure}
% TODO: add source transaction for all scenarios
\newpage
\begin{table}
    \centering
    \resizebox{\textwidth}{!}{
        \begin{tabular}{|cccccc|}
            \hline
              & \textbf{Scenario 1} & \textbf{Scenario 2} & \textbf{Scenario 3} & \textbf{Scenario 4} & \textbf{Scenario 5}\\
             \textbf{Average Throughput (msgs/sec)} & 27,090 & 50,712 & 60,952 & 110,192 & 128,496 \\
             \textbf{Median Throughput (msgs/sec)} & 28,749  & 55,827 & 64,379 & 124,469 & 134,596 \\ 
             \textbf{Maximum Throughput (msgs/sec)} & 33,968 & 71,576 & 85,824 & 147,907 & 201,680 \\
            \hline
        \end{tabular}
    }
    \caption{Kafka Replication Message Throughput Results (Source Connector)}
    \label{tab:kafka:messagethroughputsource}
\end{table}

\begin{table}
    \centering
    \resizebox{\textwidth}{!}{
        \begin{tabular}{|cccccc|}
            \hline
              & \textbf{Scenario 1} & \textbf{Scenario 2} & \textbf{Scenario 3} & \textbf{Scenario 4} & \textbf{Scenario 5}\\
             \textbf{Average Throughput (msgs/sec)} & 30,243 & 52,530 & 59,263 & 101,608 & 117,870 \\
             \textbf{Median Throughput (msgs/sec)} & 31,650 & 54,300 & 66,700 & 119,000 & 128,500 \\ 
             \textbf{Maximum Throughput (msgs/sec)} & 33,900 & 65,500 & 74,100 & 135,000 & 143,000 \\
            \hline
        \end{tabular}
    }
    \caption{Kafka Replication Message Throughput Results (Broker)}
    \label{tab:kafka:messagethroughputbroker}
\end{table}

\begin{table}
    \centering
    \resizebox{\textwidth}{!}{
        \begin{tabular}{|cccccc|}
            \hline
              & \textbf{Scenario 1} & \textbf{Scenario 2} & \textbf{Scenario 3} & \textbf{Scenario 4} & \textbf{Scenario 5}\\
             \textbf{Average Throughput (msgs/sec)} & 15,757 & 33,755 & 43,388 & 54,058 & 96,696 \\
             \textbf{Median Throughput (msgs/sec)} & 16,259 & 36,078 & 46,695 & 62,049 & 117,736 \\ 
             \textbf{Maximum Throughput (msgs/sec)} & 17,804 & 41,979 & 58,667 & 75,195 & 156,512 \\
            \hline
        \end{tabular}
    }
    \caption{Kafka Replication Message Throughput Results (Sink Connector)}
    \label{tab:kafka:messagethroughputsink}
\end{table}

\begin{table}[htbp]
    \centering
    \resizebox{\textwidth}{!}{
        \begin{tabular}{|cccccc|}
            \hline
              & \textbf{Scenario 1} & \textbf{Scenario 2} & \textbf{Scenario 3} & \textbf{Scenario 4} & \textbf{Scenario 5}\\
             \textbf{Average Throughput (transactions/sec)} & 14 & 24 & 27 & 49 & 58 \\
             \textbf{Median Throughput (transactions/sec)} & 15 & 26 & 30 & 55 & 60 \\ 
             \textbf{Maximum Throughput (transactions/sec)} & 20 & 35 & 52 & 74 & 99 \\
            \hline
        \end{tabular}
    }
    \caption{Kafka Replication Source Connector Transaction Throughput Results}
    \label{tab:kafka:transactionthroughput}
\end{table}
