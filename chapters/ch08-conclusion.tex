\chapter{Conclusion}
\label{ch08:conclusion}
The main goal of this thesis was to compare the performance of two Adabas replication solutions to a relational database: the Software GmbH's non-parallelized Event Replicator Target Adapter and a custom parallelization-capable replication solution based on Kafka Connect. The resulting research question was \textbf{"How does the performance of a Kafka Connect-based replication pipeline for Adabas on mainframe compare to the Adabas Event Replicator Target Adapter?"}, with the hypothesis that utilizing Kafka's parallelization capabilities would allow the Kafka replication solution to out-perform \ac{ART}. It was also hypothesized that in a scenario where no parallelism was implemented in the Kafka replication solution, the performance would be approximately the same or even worse compared to \ac{ART}. This was expected due to additional complexity from the distributed nature of the Kafka replication solution. An experiment was performed to explore this research question by measuring the performance of both solutions when replicating an Adabas file. To also the explore the effect of parallelization and distributedness on performance, the Kafka Connect solution was run with five different configuration scenarios. The configurations ranged from no parallelized processing at all to high parallelization with a workload distribution on multiple workers.
% todo below: add more discussion points to summary?

The results of the experiment revealed that the performance of the Kafka Connect-based replication pipeline was significantly higher than that of \ac{ART}, even when no parallelization was implemented. Kafka's parallelization capabilities allowed for performance to be improved even further, showing that Kafka's scalable nature is of great benefit when a high-performance replication solution is required. The distribution of the workload among multiple workers was also found to improve performance. However, increased scaling of the replication pipeline led to the highlighting of performance bottlenecks, potentially related to the \ac{REPTOR} messaging system and the Postgres database. Solutions for improving these performance bottlenecks, as well as considerations for further scalability, were outlined in this thesis. The limitations of the Kafka-based pipeline, such as its lack of transactional support, were also discussed and taken into account. Nevertheless, the lack of transactional support of the \ac{JDBC} sink connector was not expected to have a significant effect on the overall performance when compared to \ac{ART}. Another limitation, namely the lack of strong consistency guarantees, was seen as a necessary trade-off for improved performance. Additional enhancements to the replication pipeline, such as the implementation of Kafka Streams, were outlined to improve consistency if required.

Although Adabas is a legacy system, it remains an extremely vital component for many businesses worldwide. Integrating it with modern systems without restricting its high performance capabilities remains an important step in maintaining its relevance. The Kafka Connect replication solution offers scalable performance, integration with heterogeneous systems, and much more.
% The custom source connector can be easily modified to suit various use cases, and the modified \ac{JDBC} sink connector can be replaced with any available sink connector or a custom implementation.

\section{Future Work}
This thesis offers an overview of the fundamentals of both Adabas replication solutions, as well as the considerations that occurred during the development and from the evaluation of the experiment results. There is no current research on the performance of \ac{ART}, and not enough research on the performance of the Kafka Connect framework specifically. This led to a variety of unexplored areas in both technologies, which were highlighted for future research in the discussion. Thoroughly exploring the performance bottlenecks observed in the Kafka Connect replication solution is of high priority. It would be of additional interest to investigate factors such as the effect of complex Adabas data structures and transactionality on performance. Further research is also required to explore the optimal configuration of the Kafka cluster for even higher performance in Adabas replication.

This work lays a foundation for future exploration into optimizing and scaling Adabas data replication to modern systems with the use of Kafka, while maintaining the high performance benefits that Adabas provides.