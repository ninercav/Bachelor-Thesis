%*******************************************************
% Abstract in German
%*******************************************************
\begin{otherlanguage}{ngerman}
	\pdfbookmark[0]{Zusammenfassung}{Zusammenfassung}
	\chapter*{Zusammenfassung}
    Der Event Replicator Target Adapter für Adabas für z/OS ermöglicht die Replikation von Adabasdateien in moderne Systeme wie relationale Datenbanken, um das legacy prä-relationale Adabas zu erweitern. Allerdings stößt der Target Adapter auf Leistungsprobleme, da er zum Beispiel keine parallele Verarbeitung von Replikationsereignissen unterstützt. Das Kafka Connect Framework wurde verwendet, um eine Alternative zu dem Target Adapter zu entwickeln, der insbesondere diese Schwäche des Target Adapters adressiert.
    Diese Arbeit untersucht die wissenschaftliche Fragestellung: \textbf{„Wie ist die Performanz einer Kafka Connect-basierten Replikationspipeline für Adabas auf dem Mainframe im Vergleich zum Adabas Event Replicator Target Adapter?“}. Die beiden Replikationsansätze wurden verglichen indem ihre Leistungsergebnisse evaluiert wurden. Die Leistung wurde mithilfe von JMX-Metriken und verteiltem Tracing mit Zipkin gemessen. Die Experimente umfassten mehrere unterschiedliche Konfigurationsszenarien des Kafka Clusters im Vergleich zum nichtparallelisierbaren Target Adapter. Diese Szenarien wurden konzipiert, um den Einfluss der Parallelisierung und Verteilung auf die Leistung des Kafka-basierten Prototyps zu untersuchen.
    Die Hypothese war, dass die Leistung der Kafka-basierten Pipeline den Target Adapter übertreffen würde, wenn ihre verteilte Parallelverarbeitung optimal genutzt wird. Falls sie jedoch auch als nicht-parallele Anwendung ausgeführt würde, wurde erwartet, dass die Leistung gleich oder sogar schlechter wäre, da von den zusätzlichen Komponenten höhere Komplexität und Latenz erwartet wurde. Die Experimentergebnisse zeigten den positiven Effekt der Parallelisierung auf die Performanz des Kafka-basierten Prototyps, wobei die Distribution ebenfalls die Performanzsteigerung positiv beeinflusst hat. Der Kafka-basierte Prototyp übertraf den Target Adapter sowohl im Single-Threaded Szenario als auch in den hochparallelisierten Szenarien.
    Diese Arbeit ist von Bedeutung, da sie eine Forschungslücke zur Modernisierung prärelationaler Datenbanken wie Adabas mit modernen Ansätzen adressiert. Kafka wurde eingesetzt, um eine parallele Verarbeitung sowie eine konfigurierbare Integration mit verschiedenen Systemen zu ermöglichen. Diese Integration ermöglicht die Skalierbarkeit, Widerstandsfähigkeit und Leistung von Adabas Datenreplikation und verbessert die Anbindung von Adabas an moderne, heterogene Systeme.

\end{otherlanguage}
