%*******************************************************
% Abstract in English
%*******************************************************
\pdfbookmark[0]{Abstract}{Abstract}


\begin{otherlanguage}{american}
	\chapter*{Abstract}
    % shorten first sentences so it's not the same as intro - immediately go into what this thesis will explore
	To modernize the legacy pre-relational Adabas, the Event Replicator Target Adapter for Adabas on z/OS enables replication to modern systems, including relational databases. However, it faces performance challenges due to factors such as no support for parallel processing of replication events. The Kafka Connect framework was used to create a prototype to address the Target Adapter's drawbacks, focusing mainly on the lack of parallelization. This research explores the question \textbf{"How does the performance of a Kafka Connect-based replication pipeline for Adabas on mainframe compare to the Adabas Event Replicator Target Adapter?"}. The two replication approaches will be compared by evaluating their performance differences. The performance will be measured using JMX metrics and distributed tracing with Zipkin. Experiments will include various configuration scenarios of the Kafka cluster against the single-process Target Adapter to see how the degree of parallelization and distributedness affects the Kafka-based prototype's performance.
    % and simulated failures, such as worker node crashes, to observe recovery and rebalancing capabilities.
    The hypothesis is that the Kafka pipeline will outperform the Target Adapter if its distributed parallel processing capabilities are properly leveraged. If it is instead run as a single-thread application like the Target Adapter, then the performance will be equal or even worse, due to higher complexity and latency incurred by the multiple components. This research proves significant because it addresses a gap in the research concerning the modernization of pre-relational databases such as Adabas with modern approaches. Kafka is employed to facilitate parallel processing and a configurable integration with various systems. Such integration enables enhanced scalability, performance, and resilience in data replication pipelines, as well as improving integration with heterogeneous systems.
\end{otherlanguage}
